\documentclass[bachelor]{thesis-uestc}

\title{基于语法树的文档级关系预测算法设计与验证}{Design and Verification of Document Level Relation Prediction}
\author{朱旭东}{Xudong Zhu}
\advisor{康昭\chinesespace 副教授}{Dr. Zhao Kang}
\school{计算机科学与工程学院(网络空间安全学院)}{School of Computer Science and Engineering}
\major{计算机科学与技术}{Computer Science and Technology}
\studentnumber{2020080902004}

\begin{document}

\makecover

\begin{chineseabstract}
    为了适应日益增长的宽带信号和非线性系统的工程应用,用于分析瞬态电磁散射问题的时域积分方程方法研究日趋活跃。本文以时域积分方程时间步进算法及其快速算法为研究课题,重点研究了时间步进算法的数值实现技术、后时稳定性问题以及两层平面波算法加速计算等,主要研究内容分为四部分。

    \chinesekeyword{时域电磁散射,时域积分方程,时间步进算法,后时不稳定性,时域平面波算法}
\end{chineseabstract}

\begin{englishabstract}
    With the widespread engineering applications ranging from broadband signals and non-linear systems, time-domain integral equations (TDIE) methods for analyzing transient electromagnetic scattering problems are becoming widely used nowadays. TDIE-based marching-on-in-time (MOT) scheme and its fast algorithm are researched in this dissertation, including the numerical techniques of MOT scheme, late-time stability of MOT scheme, and two-level PWTD-enhanced MOT scheme. The contents are divided into four parts shown as follows.

    \englishkeyword{Time-domain Electromagnetic Scattering, Time-domain Integral Equation, Marching-on In-time (MOT) Scheme, Late-time Instability, Plane Wave Time-domain (PWTD) Algorithm}
\end{englishabstract}

\thesistableofcontents

\chapter{绪\hspace{6pt}论}


\thesisacknowledgement
在攻读博士学位期间,首先衷心感谢我的导师XXX教授

\thesisappendix


% Uncomment to list all the entries of the database.

\thesisbibliography{reference}

%
% Uncomment following codes to load bibliography database with native
\bibliography command.

\bibliographystyle{thesis-uestc}
\bibliography{reference}


\thesisaccomplish{publications}

\thesistranslationoriginal
\section{The OFDM Model of Multiple Carrier Waves}

\thesistranslationchinese
\section{基于多载波索引键控的正交频分多路复用系统模型}

\end{document}
